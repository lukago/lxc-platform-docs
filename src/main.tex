\documentclass[twoside,a4paper,12pt]{report} %draft
\pdfminorversion=7
%\usepackage[lmargin=100]{geometry}
%\usepackage{showframe}
\usepackage[T1]{fontenc}
\usepackage[utf8]{inputenc}
\usepackage{polski}
\usepackage{pdfpages}
\usepackage{biblatex}
\usepackage{fancyhdr}
\usepackage{listings}
\usepackage{graphicx}
\usepackage{anysize}
\usepackage{float}
\usepackage[allcolors=blue]{hyperref}

% for margins left, right top botto
\marginsize{3cm}{2.5cm}{2.5cm}{2.5cm}
\setlength{\headheight}{14.5pt}
\addbibresource{biblio.bib}
\pagestyle{fancy}
\graphicspath{ {./figures/} }
\renewcommand{\listfigurename}{Spis rysunków}
\renewcommand{\listtablename}{Spis tabel}
\renewcommand{\lstlistingname}{Fragment programu}
\renewcommand{\lstlistlistingname}{Spis fragmentów programów}

\begin{document}
  \includepdf[pages=-]{includes/first-page.pdf}
  \newpage\leavevmode\thispagestyle{empty}\newpage
  %\addcontentsline{toc}{chapter}{Spis treści}
  \tableofcontents

  \chapter{Wstęp}
\label{ch:wstęp}
\section{Cel}
\label{sec:cel}
Celem pracy jest zbadanie możliwości, jakie daje zestaw narzędzi Java EE
w zakresie tworzenia systemu internetowego~\cite{einstein},
na przykładzie internetowego systemu wspomagającego przeprowadzanie ankiet. \\
Wybór technologii został podyktowany szczegółową analizą popularnych języków programowania,
wnioski z której przedstawione zostały w tabeli~\ref{tab:porownanie_jezykow}.
Architektura tworzonego systemu przedstawiona jest na rysunku~\ref{fig:architektura_warstwowa}.

\begin{table}[H]
  \centering
  \begin{tabular}{|c|c|c|}
    \hline
    & Java & C++ \\
    \hline
    Opis & FAJNA& SŁABY \\
    \hline
  \end{tabular}
  \caption{Porównanie języków programowania.}
  \label{tab:porownanie_jezykow}
\end{table}

\begin{figure}[H]
  \centering \includegraphics[width=0.9\textwidth]{figures/layers-arch}
  \caption{Architektura warstwowa aplikacji.}
  \label{fig:architektura_warstwowa}
\end{figure}

\cleardoublepage

\subsection{Podsekcja}
\label{subsec:podsekcja}
Cel udało się osiągnąć przy użyciu programu, którego kod przedstawiony jest poniżej.

\begin{lstlisting}[caption={Wyświetlenie "SPOKO!"},captionpos=b]
  System.out.println("SPOKO!");
\end{lstlisting}

  \printbibliography[
  heading=bibintoc,
  title={Bibliografia}
  ]

  \listoffigures
  \addcontentsline{toc}{chapter}{Spis rysunków}

  \listoftables
  \addcontentsline{toc}{chapter}{Spis tabel}

  \lstlistoflistings
  \addcontentsline{toc}{chapter}{Spis fragmentów programów}

\end{document}