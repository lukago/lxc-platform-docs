\chapter{Technologie i narzędzia}
\label{ch:technologie i narzędzia}

W poprzednim rozdziale przedstawiono spis użytych technologii.
Ten rozdział skupia się na bardziej szczegółowym opisaniu tych narzędzi, tj.\ zasady ich działania
oraz w jaki sposób zostały użyte w projekcie pracy inżynierskiej.

\section{Apache Maven}
\label{sec:apache maven}

Apache Maven jest narzędziem automatyzującym proces budowania aplikacji na platformie JVM\@.
Sposób budowania aplikacji jest opisany w plikach XML\@.
Dodatkowo możemy w nich skonfigurować właściwości aplikacji oraz w prosty sposób zarządzać
zależnościami.
Zależności Mavena mogą być pobrane zarówno z lokalnego jak i zdalnego repozytorium.
Poza oficjalnym zdalnym repozytorium Mavena, możemy konfigurować również inne repozytoria zdalne,
dostarczające własne moduły.
Cechą wyróżniającą Mavena od innych tego typu narzędzi jest strategia posługiwania się przyjętymi
konwencjami, zamiast używania własnej konfiguracji (ang.\ \textit{Convention Over Configuration}).
Maven został wykorzystany w projekcie jako narzędzia do budowania aplikacji serwerowej.

\section{Docker}
\label{sec:docker}

Docker to narzędzie ułatwiające proces wdrażania aplikacji przy użyciu kontenerów, czyli
wirtualizacji na poziomie systemu operacyjnego.
Kontenery Docker zawierają aplikację wraz z jej konfiguracją i wszystkimi zależnościami.
W założeniu kontenery Docker są izolowanie od siebie oraz nie zachowują swojego stanu po wyłączeniu.
Docker poza konteneryzacją dostarcza szereg dodatkowych mechanizmów.
Jednym z nich są wolumeny, które umożliwiają dostęp do plików na systemie operacyjnym gospodarza,
co umożliwia, np.\ kontenerowi z bazą danych, permanentnie zapisywać dane na dysku.
Docker został użyty podczas implementacji projektu jako narzędzie do uruchamiania aplikacji
serwerowej oraz klienckiej.

\section{Guava}
\label{sec:guava}

Magnum Opus

\section{Hibernate}
\label{sec:hibernate}

Magnum Opus

\section{JJWT}
\label{sec:jjwt}

Magnum Opus

\section{LXD}
\label{sec:lxd}

Magnum Opus

\section{Material-UI}
\label{sec:meterialui}

Magnum Opus

\section{ModelMapper}
\label{sec:modelmapper}

Magnum Opus

\section{Node.js}
\label{sec:nodejs}

Magnum Opus

\section{OpenJDK}
\label{sec:openjdk}

OpenJDK to zestaw narzędzi deweloperskich dla języka Java (ang.\ \textit{Java Development Kit}).
Najważniejsze narzędzia wchodzące w skład OpenJDK to maszyna wirtualna
Javy (w skrócie JVM, ang.\ \textit{Java Virtual Machine}), biblioteka standardowa,
oraz kompilator \textit{javac}.
Język Java jest wysokopoziomowym językiem programowania wspierającym wiele paradygmatów
programowania, przede wszystkim paradygmat programowania obiektowego, ale również np.\ funkcyjnego.
Jest to język silnie i statycznie typowany, kompilowany do kodu bajtowego (ang. \textit{bytecode}).
Kod bajtowy jest uruchamiany przez maszynę wirtualną Javy i tłumaczony na kod kod binarny
odpowiadający architekturze danego systemu, na którym maszyna wirtualna Javy jest zainstalowania.
Dzięki temu zastosowanie maszyny wirtualnej umożliwiło programom napisanym w języku Java
na uniezależnienie się od architektury.
Dodatkowo maszyna wirtualna Javy optymalizuje czas wykonania programu dzięki zastosowania techniki
JIT (ang.\ \textit{Just-In-Time compilation}), czyli kompilacji fragmentu kodu bajtowego do
kodu maszynowego bezpośrednio przez wykonaniem danego fragmentu kodu.
Kolejną zaletą języka Java jest obszerna biblioteka standardowa, dodająca szereg klas wpierających
np.\ programowanie współbieżne i sieciowe czy operacje wejścia-wyjścia.
Dzięki ww.\ aspektom język Java dobrze nadaje się do pisania złożonych aplikacji internetowych.
Z tego powodu został wybrany jako język implementacji części serwerowej projektu pracy
inżynierskiej.

\section{Spring Framework}
\label{sec:spring}

Magnum Opus

\section{SpringFox}
\label{sec:springfox}

Magnum Opus

\section{SSHJ}
\label{sec:ssj}

Magnum Opus

\clearpage