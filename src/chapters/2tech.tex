\chapter{Technologie i narzędzia}
\label{ch:technologie i narzędzia}

W poprzednim rozdziale przedstawiono spis użytych technologii.
Ten rozdział skupia się na bardziej szczegółowym opisaniu tych narzędzi, tj.\ zasady ich działania
oraz w jaki sposób zostały użyte w projekcie pracy inżynierskiej.

\section{Apache Maven}
\label{sec:apache maven}

Apache Maven jest narzędziem automatyzującym proces budowania aplikacji na platformie JVM\@.
Sposób budowania aplikacji jest opisany w plikach XML\@.
Dodatkowo możemy w nich skonfigurować właściwości aplikacji oraz w prosty sposób zarządzać
zależnościami.
Zależności Maven mogą być pobrane zarówno z lokalnego jak i zdalnego repozytorium.
Poza oficjalnym zdalnym repozytorium Maven, możemy konfigurować również inne repozytoria zdalne,
dostarczające własne moduły.
Cechą wyróżniającą projekt Maven od innych tego typu narzędzi jest strategia posługiwania się przyjętymi
konwencjami, zamiast używania własnej konfiguracji (ang.\ \textit{Convention Over Configuration}).
Maven został wykorzystany w projekcie jako narzędzia do budowania aplikacji serwerowej.

\section{Docker}
\label{sec:docker}

Docker to narzędzie ułatwiające proces wdrażania aplikacji przy użyciu kontenerów, czyli
wirtualizacji na poziomie systemu operacyjnego.
Kontenery Docker zawierają aplikację wraz z jej konfiguracją i wszystkimi zależnościami.
W założeniu kontenery Docker są izolowanie od siebie oraz nie zachowują swojego stanu po wyłączeniu.
Docker poza konteneryzacją dostarcza szereg dodatkowych mechanizmów.
Jednym z nich są wolumeny, które umożliwiają dostęp do plików na systemie operacyjnym gospodarza,
co umożliwia, np.\ kontenerowi z bazą danych, permanentnie zapisywać dane na dysku.
Docker został użyty podczas implementacji projektu jako narzędzie do uruchamiania aplikacji
serwerowej oraz klienckiej.

\section{Guava}
\label{sec:guava}

Guava to biblioteka autorstwa Google zawierająca wiele klas użytkowych, jest to rozszerzenie biblioteki standardowej
języka Java.
Z tego powodu jest to jedna z najpopularniejszych zewnętrznych bibliotek używanych w projektach opartych na JVM\@.
W projekcie pracy inżynierskiej również znalazła swoje zastosowanie np.\ podczas implementacji metod \textit{equals}
i \textit{hashCode}.

\section{Hibernate}
\label{sec:hibernate}

Hibernate dostarcza mechanizm ORM (ang.\ \textit{Object Relational Mapping}), czyli mapowania relacyjno-obiektowego
dla platformy JVM\@.
Umożliwia to operowanie na encjach bazy relacyjnej za pomocą odpowiadających im obiektów zdefiniowanych w
języku Java.
Biblioteka Hibernate implementuje specyfikacje JPA (ang.\ \textit{Java Persistence API}), dzięki czemu może być
z łatwością wykorzystana w każdej aplikacji wykorzystującej JPA np.\ aplikacjach Java EE\@.
Warstwę danych części serwerowej projektu pracy inżynierskiej, stanowi relacyjna baza danych.
Ponieważ w części serwerowej projektu pracy inżynierskiej, do komunikacji z relacyjną bazą danych
wykorzystano interfejs JPA, Hibernate został użyty jako główna implementacja tego interfejsu.

\section{JJWT}
\label{sec:jjwt}

JJWT (inaczej Java JWT) to biblioteka napisana w języku Java, służąca do tworzenia i weryfikacji
JWT (ang.\ \textit{JSON Web Token}).
JJWT jest zgodna ze specyfikacją RFC odnośnie JWT\@.
W projekcie pracy inżynierskiej aplikacja serwerowa nie przechowuje stanu o sesji użytkowników.
Z tego powodu do autoryzacji i uwierzytelniania użytkowników wykorzystano JWT, a biblioteka JJWT ułatwiła implementację
tych mechanizmów.

\section{LXD i kontenery Linux}
\label{sec:lxd}

LXD to aplikacja do zarządzania kontenerami Linux.
Aplikacja LXD jest zaimplementowana w języku Go, jednak dostarcza interfejs konsolowy, to właśnie za jego pomocą została
zaimplementowana integracja z kontenerami w projekcie pracy inżynierskiej.
LXD bazuje na obrazach, dostarczając wiele wcześniej przygotowanych obrazów dla szerokiej rangi dystrybucji Linuxa.
Aplikacja LXD tworzy REST API przez które wykonywane są wszystkie operacje na kontenerach, nawet używając aplikację
za pomocą linii komend, każda komenda wywołuje w swojej implementacji zapytanie na ww.\ interfejs REST\@.
LXD nie jest alternatywą dla LXC, jest to aplikacja wykorzystująca LXC i dostarczająca dodatkowe funkcjonalności.

Najważniejsze z dodatkowych funkcjonalności LXD to:

\begin{itemize}
  \item Przechowywanie konfiguracji i stanu kontenera, w relacyjnej bazie
  danych, zamiast w zwykłych folderach, jak ma to miejsce przy wykorzystaniu LXC\@.
  \item Mechanizmy do prostego zarządzania sieciami kontenerów i ich integracja z siecią
  gospodarza np.\ za pomocą mostów sieciowych (ang.\ \textit{bridge}).
  \item Przekazywanie urządzeń do kontenerów (ang.\ \textit{device passthrough}) np.\ urządzeń USB, GPU, NIC\@.
  \item Mechanizmy do kontrolowania zasobów kontenerów takich jak wykorzystanie procesora, pamięci RAM czy przestrzeni
  dyskowej.
\end{itemize}

\section{ModelMapper}
\label{sec:modelmapper}

ModelMapper to biblioteka ułatwiająca mapowanie podobnych obiektów między sobą.
W wielu aplikacjach często pojawiają się obiekty bardzo podobne do siebie (np.\ posiadające te same nazwy i typy pól),
lecz pochodzą z równych modułów i używane są w innych kontekstach.
Wykorzystując ModelMapper konwertowanie podobnych obiektów dzieje się automatycznie, nie ma potrzeby pisania własnych
funkcji mapujących.
W projekcie pracy inżynierskiej ModelMapper został wykorzystany na mapowanie pomiędzy obiektami reprezentującymi
encje w bazie danych na obiekty typu DTO (ang.\ \textit{Data Transfer Object}).

\section{Node.js i Npm}
\label{sec:nodejs}

Magnum Opus

\section{OpenJDK}
\label{sec:openjdk}

OpenJDK to zestaw narzędzi deweloperskich dla języka Java (ang.\ \textit{Java Development Kit}).
Najważniejsze narzędzia wchodzące w skład OpenJDK to maszyna wirtualna
Javy (w skrócie JVM, ang.\ \textit{Java Virtual Machine}), biblioteka standardowa,
oraz kompilator \textit{javac}.
Język Java jest wysokopoziomowym językiem programowania wspierającym wiele paradygmatów
programowania, przede wszystkim paradygmat programowania obiektowego, ale również np.\ funkcyjnego.
Jest to język silnie i statycznie typowany, kompilowany do kodu bajtowego (ang. \textit{bytecode}).
Kod bajtowy jest uruchamiany przez maszynę wirtualną Javy i tłumaczony na kod kod binarny
odpowiadający architekturze danego systemu, na którym maszyna wirtualna Javy jest zainstalowania.
Dzięki temu zastosowanie maszyny wirtualnej umożliwiło programom napisanym w języku Java
na uniezależnienie się od architektury.
Dodatkowo maszyna wirtualna Javy optymalizuje czas wykonania programu dzięki zastosowania techniki
JIT (ang.\ \textit{Just-In-Time compilation}), czyli kompilacji fragmentu kodu bajtowego do
kodu maszynowego bezpośrednio przez wykonaniem danego fragmentu kodu.
Kolejną zaletą języka Java jest obszerna biblioteka standardowa, dodająca szereg klas wpierających
np.\ programowanie współbieżne i sieciowe czy operacje wejścia-wyjścia.
Dzięki ww.\ aspektom język Java dobrze nadaje się do pisania złożonych aplikacji internetowych.
Z tego powodu został wybrany jako język implementacji części serwerowej projektu pracy
inżynierskiej.

\section{PostgreSQL}
\label{sec:postgres}

Magnum Opus

\section{Project Reactor}
\label{sec:reactor}

Magnum Opus

\section{React i Material-UI}
\label{sec:meterialui}

Magnum Opus

\section{Spring Framework}
\label{sec:spring}

Magnum Opus

\section{SpringFox}
\label{sec:springfox}

Magnum Opus

\section{SSHJ}
\label{sec:ssj}

SSHJ to biblioteka napisana w języku Java ułatwiająca implementacje funkcjonalności wykorzystujących protokół SSH\@.
Dostarcza między innymi możliwości autoryzacji hasłem i kluczem publicznym, przekazywanie
portów (ang.\ \textit{port forwarding}) oraz kanały do wykonywania zdalnych poleceń z powłoki systemu operacyjnego.
W pracy inżynierskiej SSHJ wykorzystano do komunikacji ze zdalnym serwerem odpowiedzialnym za zarządzanie kontenerami
Linux.

\clearpage