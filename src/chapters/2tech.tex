\chapter{Technologie i narzędzia}
\label{ch:technologie_narzedzia}

W poprzednim rozdziale przedstawiono spis użytych technologii.
Ten rozdział skupia się na bardziej szczegółowym opisaniu tych narzędzi, opisuje zasady
ich działania oraz w jaki sposób zostały użyte w projekcie pracy inżynierskiej.

\section{Apache Maven}
\label{sec:apache_maven}

Apache Maven jest narzędziem automatyzującym proces budowania aplikacji na platformie JVM\@.
Sposób budowania aplikacji jest definiowany w plikach XML\@.
Dodatkowo można w nich skonfigurować właściwości aplikacji oraz w prosty sposób zarządzać
zależnościami.
Zależności Maven mogą być pobrane zarówno z lokalnego jak i zdalnego repozytorium.
Poza oficjalnym zdalnym repozytorium Maven, można konfigurować również inne repozytoria zdalne,
dostarczające własne moduły.
Cechą wyróżniającą projekt Maven od innych tego typu narzędzi, jest strategia posługiwania
się przyjętymi konwencjami, zamiast używania własnej konfiguracji
(ang.\ \textit{Convention Over Configuration})~\cite{mavencookbook}.
Maven został wykorzystany w projekcie pracy inżynierskiej jako narzędzie do budowania
aplikacji serwerowej.

\section{Docker}
\label{sec:docker}

Docker to narzędzie ułatwiające proces wdrażania aplikacji przy użyciu wirtualizacji na poziomie
systemu operacyjnego czyli kontenerów.
Kontenery Docker zawierają aplikację wraz z jej konfiguracją i wszystkimi zależnościami.
W założeniu kontenery Docker są izolowanie od siebie oraz nie zachowują swojego stanu po wyłączeniu.
Docker poza konteneryzacją dostarcza szereg dodatkowych mechanizmów.
Jednym z nich są wolumeny, które umożliwiają dostęp do plików na systemie operacyjnym gospodarza,
co umożliwia, np.\ kontenerowi z bazą danych, permanentnie zapisywać dane
na dysku~\cite{dockerpractice}.
Docker został użyty podczas implementacji projektu jako narzędzie do uruchamiania aplikacji
serwerowej oraz klienckiej.

\section{Guava}
\label{sec:guava}

Guava to biblioteka autorstwa Google zawierająca wiele klas użytkowych, jest niejako rozszerzeniem
biblioteki standardowej języka Java.
Guava dodaje m.in.\ nowe typy kolekcji, klasy wspierające współbieżność, łańcuchy znaków, operowanie
na strumieniach wejścia i wyjścia~\cite{guavastarted}.
Z tego powodu jest to jedna z najpopularniejszych zewnętrznych bibliotek używanych w projektach
opartych na JVM\@.
W projekcie pracy inżynierskiej również znalazła swoje zastosowanie np.\ podczas implementacji
metod \textit{equals}i \textit{hashCode}.

\section{Hibernate}
\label{sec:hibernate}

Hibernate dostarcza mechanizm ORM (ang.\ \textit{Object Relational Mapping}), czyli mapowania
relacyjno-obiektowego dla platformy JVM\@.
Umożliwia to operowanie na encjach bazy relacyjnej za pomocą odpowiadających im obiektów
zdefiniowanych w języku Java.
Biblioteka Hibernate implementuje specyfikację JPA (ang.\ \textit{Java Persistence API}),
dzięki czemu może być z łatwością wykorzystana w każdej aplikacji wykorzystującej
JPA, np.\ aplikacjach Java EE~\cite{hibernatejpa}.
Warstwę danych części serwerowej projektu pracy inżynierskiej, stanowi relacyjna baza danych.
Ponieważ w części serwerowej projektu pracy inżynierskiej do komunikacji z relacyjną bazą danych
wykorzystano interfejs JPA, Hibernate został użyty jako główna implementacja
tego interfejsu.

\section{JJWT}
\label{sec:jjwt}

JJWT (inaczej Java JWT) to biblioteka napisana w języku Java, służąca do tworzenia i weryfikacji
JWT (ang.\ \textit{JSON Web Token}).
JJWT jest zgodna ze specyfikacją RFC odnośnie JWT~\cite{jjwtdocs}.
W projekcie pracy inżynierskiej aplikacja serwerowa nie przechowuje stanu o sesji użytkowników.
Z tego powodu do autoryzacji i uwierzytelniania użytkowników wykorzystano JWT, a biblioteka JJWT
ułatwiła implementację tych mechanizmów.

\section{LXD}
\label{sec:lxd}

LXD to aplikacja do zarządzania LXC czyli kontenerami Linux.
Aplikacja LXD jest zaimplementowana w języku Go, jednak dostarcza interfejs konsolowy oraz
interfejs sieciowy, za którego pomocą możliwa jest integracja z programem z poziomu innych języków.
LXD bazuje na obrazach, dostarczając wiele wcześniej przygotowanych obrazów dla szerokiej rangi
dystrybucji systemów operacyjnych GNU/Linux.

Aplikacja LXD tworzy REST API przez które wykonywane są wszystkie operacje na kontenerach,
nawet używając aplikację za pomocą linii komend, każda komenda wywołuje w swojej implementacji
zapytanie na ww.\ interfejs REST\@.
LXD nie jest alternatywą dla LXC, jest to aplikacja wykorzystująca LXC i dostarczająca dodatkowe
funkcjonalności~\cite{lxdpractical}.
Najważniejsze z dodatkowych funkcjonalności LXD to:

\begin{itemize}
  \item Przechowywanie konfiguracji i stanu kontenera w bazie
  danych, zamiast w zwykłych folderach, jak ma to miejsce przy wykorzystaniu LXC\@.
  \item Mechanizmy do prostego zarządzania sieciami kontenerów i ich integracja z siecią
  gospodarza np.\ za pomocą mostów sieciowych (ang.\ \textit{bridge}).
  \item Prostsze przekazywanie urządzeń do kontenerów (ang.\ \textit{device passthrough}),
  np.\ urządzeń USB, GPU, NIC\@.
  \item Mechanizmy do kontrolowania zasobów kontenerów takich jak wykorzystanie procesora,
  pamięci RAM czy przestrzeni dyskowej.
\end{itemize}

\section{ModelMapper}
\label{sec:modelmapper}

ModelMapper to biblioteka ułatwiająca mapowanie podobnych obiektów między sobą.
W wielu aplikacjach często pojawiają się obiekty bardzo podobne do siebie,
np.\ posiadające te same nazwy i typy pól, lecz pochodzą z równych modułów i używane są w
innych kontekstach.
Wykorzystując ModelMapper konwertowanie podobnych obiektów dzieje się automatycznie,
nie ma potrzeby pisania własnych funkcji mapujących~\cite{modelmapperdocs}.
W projekcie pracy inżynierskiej ModelMapper został wykorzystany na mapowanie pomiędzy obiektami
reprezentującymi encje w bazie danych na obiekty typu
DTO (ang.\ \textit{Data Transfer Object}).

\section{Node.js i Npm}
\label{sec:nodejs}

JavaScript jest językiem o słabym i dynamicznym typowaniu, używanym głównie na stronach WWW
i interpretowany przez przeglądarki internetowe.
Node.js to środowisko uruchomieniowe dla języka JavaScript, stworzone na bazie silnika Chrome V8.
Umożliwiło to używanie języka nie tylko na stronach WWW, ale również w pełnoprawnych aplikacjach,
nieprzeznaczonych do interpretacji przez przeglądarkę.

Npm (ang.\ \textit{Node Package Manager}) został utworzony w technologii Node.js.
Jest to narzędzie, którym w prosty sposób można zarządzać zależnościami programów napisanych
napisanych w języku JavaScript.
Dodatkowo Npm udostępnia szereg innych funkcjonalności do zarządzania projektami, np.\ definiowanie
skryptów budujących i startujących tworzoną aplikację, czy zarządzanie wieloma wersjami
projektów~\cite{nodejsinaction}.

Narzędzia Node.js i Npm zostały wykorzystane przy implementacji warstwy widoku projektu pracy
inżynierskiej.
Dostarczyły one możliwość zbudowania i transformacji kodu źródłowego aplikacji klienckiej
do finalnej wersji interpretowanej przez przeglądarkę internetową.

\section{OpenJDK}
\label{sec:openjdk}

OpenJDK to zestaw narzędzi deweloperskich dla języka Java (ang.\ \textit{Java Development Kit}).
Najważniejsze narzędzia wchodzące w skład OpenJDK to maszyna wirtualna
Javy (w skrócie JVM, ang.\ \textit{Java Virtual Machine}), biblioteka standardowa,
oraz kompilator \textit{javac}.

Język Java jest wysokopoziomowym językiem programowania wspierającym wiele paradygmatów
programowania, przede wszystkim paradygmat programowania obiektowego, ale również np.\ funkcyjnego.
Jest to język silnie i statycznie typowany, kompilowany do kodu bajtowego (ang. \textit{bytecode}).
Kod bajtowy jest uruchamiany przez maszynę wirtualną Javy i tłumaczony na kod kod binarny
odpowiadający architekturze danego systemu, na którym maszyna wirtualna Javy jest zainstalowana.
Dzięki temu zastosowanie maszyny wirtualnej umożliwiło programom napisanym w języku Java
na uniezależnienie się od architektury.

Dodatkowo maszyna wirtualna Javy optymalizuje czas wykonania programu dzięki zastosowania techniki
JIT (ang.\ \textit{Just-In-Time compilation}), czyli kompilacji fragmentu kodu bajtowego do
kodu maszynowego bezpośrednio przez wykonaniem danego fragmentu kodu.
Kolejną zaletą języka Java jest obszerna biblioteka standardowa, dodająca szereg klas wpierających
np.\ programowanie współbieżne i sieciowe czy operacje wejścia-wyjścia~\cite{openjdkcookbook}.

Dzięki ww.\ aspektom język Java dobrze nadaje się do pisania złożonych aplikacji internetowych.
Z tego powodu został wybrany jako język implementacji części serwerowej projektu pracy
inżynierskiej.

\section{PostgreSQL}
\label{sec:postgres}

PostgreSQL to otwartoźródłowy i bezpłatny system zarządzania relacyjną bazą danych
(ang.\ \textit{Relational Database Management System, RDBMS}), jest zgodny ze standardem SQL
oraz go rozszerza.
Jest to bardzo dojrzała platforma, rozwijana od ponad 30 lat, która dorównuje wielu
rozwiązaniom komercyjnym w aspekcie wydajności, stabilności oraz oferowanych funkcjonalności.
Umożliwia nawiązanie połączenia sieciowego z relacyjną bazą danych w modelu
klient-serwer ~\cite{psqlmasterbook}.
Dzięki ww.\ zaletom tego oprogramowania został on wybrany do zarządzania bazą danych
w części serwerowej pracy inżynierskiej.

\section{Project Reactor}
\label{sec:reactor}

Reactor to biblioteka wspierająca programowanie reaktywne na platformie JVM\@.
Integruje się z interfejsami programistycznymi dodanymi w Javie 8, takimi jak \textit{Stream},
\textit{CompletableFuture} oraz \textit{Duration}, implementuje również specyfikację
\textit{Reactive Streams}~\cite{reactordocs}.
W projekcie pracy inżynierskiej Reactor został wykorzystany w części serwerowej projektu,
wspierając implementację reaktywnego wysyłania wiadomości do gniazd.

\section{React i Material-UI}
\label{sec:react}

React to biblioteka napisana w języku JavaScript wspierająca tworzenie interfejsów użytkownika.
Główną zaletą biblioteki jest możliwość tworzenie interaktywnych interfejsów w prosty sposób.
React aktualizuje widoki automatycznie podczas zmiany stanu aplikacji, przez co użytkownik nie musi
ręcznie odświeżać widoku~\cite{reactpro}.
Material-UI to biblioteka napisana z wykorzystaniem biblioteki React.
Zawiera ona wiele gotowych i uniwersalnych komponentów, przydatnych podczas tworzenia interfejsów
użytkownika, np.\ komponenty tabel, przycisków, ikon, list czy banerów~\cite{meterialuicookbook}.
Obie te biblioteki były podstawą podczas tworzenia interfejsu użytkownika projektu pracy
inżynierskiej.

\section{Spring Framework}
\label{sec:spring}

Spring Framework to szkielet tworzenia aplikacji (ang.\ \textit{application framework})
dla platformy JVM\@.
Został stworzony jako alternatywa dla technologi EJB, oferując lżejszy i prostszy model
programowania.

Główne funkcjonalności i założenia szkieletu Spring to:

\begin{itemize}
  \item Mechanizmy wstrzykiwania zależności umożliwiające luźne powiązanie zależności.
  \item Użycie szkieletu Spring jest nieinwazyjne dzięki oparciu na obiektach POJO
  (ang.\ \textit{Plain Old Java Object}).
  \item Programowanie deklaratywne poprzez aspekty i przyjęte konwencje.
  \item Eliminowanie sekcji powtarzalnego kodu za pomocą aspektów i szablonów.
\end{itemize}

Poza ww.\ funkcjonalnościami wchodzącymi w skład tzw.\ rdzenia szkieletu Spring, istnieją również
dodatkowe moduły, które możemy dodawać do naszej aplikacji w zależności
od potrzeb~\cite{springinaction}.

W projekcie pracy inżynierskiej zostały wykorzystane następujące moduły:

\begin{itemize}
  \item \textbf{Spring Boot}.
  Moduł ułatwiający implementację podczas pracy z wszystkimi modułami
  szkieletu Spring.
  Najważniejsze z dodanych funkcjonalności to możliwość automatycznej konfiguracji za pomocą
  adnotacji w języku Java oraz startery grupujące zależności,
  co ułatwia pracę z wykorzystaniem narzędzi takich jak Maven.
  \item \textbf{Spring Data}.
  Moduł ułatwiający implementację komunikacji z warstwą danych aplikacji, w tym
  z relacyjnymi bazami danych.
  Integruje się on w interfejsami JPA, automatycznie konfiguruje Hibernate czyli
  implementacje kontraktów opisanych w JPA\@.
  Dostarcza również możliwość korzystania repozytoriów JPA za pomocą prostych interfejsów.
  \item \textbf{Spring Security}.
  Moduł odpowiedzialny za zabezpieczenia aplikacji.
  Dodaje funkcjonalności umożliwiające zabezpieczenie udostępnianych przez serwer interfejsów,
  uwierzytelnianie użytkowników i weryfikację ich poziomów dostępu czy konfiguracje
  protokołu HTTPS\@.
  \item \textbf{Spring Web}.
  Moduł umożliwiający tworzenie aplikacji internetowych.
  Dodaje możliwość konfiguracji serwera i adresów HTTP które udostępniać będzie aplikacja.
\end{itemize}

\section{SSHJ}
\label{sec:ssj}

SSHJ to biblioteka napisana w języku Java ułatwiająca implementacje funkcjonalności wykorzystujących
protokół SSH\@.
Dostarcza między innymi możliwości autoryzacji hasłem i kluczem publicznym, przekazywanie
portów (ang.\ \textit{port forwarding}) oraz kanały do wykonywania zdalnych poleceń z powłoki
systemu operacyjnego~\cite{sshjdocs}.
W pracy inżynierskiej SSHJ wykorzystano do komunikacji ze zdalnym serwerem odpowiedzialnym za
zarządzanie kontenerami Linux.

\clearpage