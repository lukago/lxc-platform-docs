\chapter{Projekt systemu}
\label{ch:projekt_systemu}

System informatyczny będący tematem pracy inżynierskiej został zrealizowany w architekturze
wielowarstwowej.
W dalszych sekcjach znajduje się szczegółowy opis każdej z warstw.
Przedstawione zostały poziomy dostępu użytkowników, modele danych,
wykorzystane wzorce projektowe, zabezpieczenia aplikacji oraz sposoby przepływu informacji
w przekroju całego systemu informatycznego.
Ogólny schemat przepływu danych pomiędzy elementami systemu został przedstawiony
na rysunku~\ref{fig:arch}.

\begin{figure}[H]
  \centering
  \includegraphics[scale=0.59]{figures/arch.png}
  \caption{Schemat przepływu danych pomiędzy elementami systemu}
  \label{fig:arch}
\end{figure}
\newpage

\section{Poziomy dostępów}
\label{sec:przypadki_uzycia}

W zaprojektowanym systemie istnieje kilka poziomów dostępów dla użytkownika.
Poszczególne poziomy dostępów to:
\begin{itemize}
  \item Poziom Gościa, jest przypisywany dla niezalogowanych użytkowników.
  \item Poziom Klienta, jest przypisywany do zalogowanych użytkowników, których konto ma przypisany
  poziom dostępu klienta.
  Umożliwia dostęp do danych związanych z danym kontem i ich ograniczoną modyfikację.
  \item Poziom Administratora, jest przypisywany do zalogowanych użytkowników, których konto ma
  przypisany poziom dostępu administratora.
  Umożliwia dostęp do danych o użytkownikach i kontenerach oraz modyfikację tych danych.
\end{itemize}

Wszystkie przypadki użycie dla poszczególnych poziomów dostępów przedstawia
tabela~\ref{tab:przypadki_uzycia}.
Litery G, K, A oznaczają odpowiednio poziomy dostępów Gościa, Klienta i Administratora.

\newpage
\begin{table}[H]
  \centering
  \caption{Macierz decyzyjna przypadków użycia}
  \begin{tabular}{|r|l|c|c|c|}
    \hline
    Lp. & Przypadek użycia & G & K & A \\
    \hline
    1 & Zaloguj & X & &  \\
    \hline
    2 & Wyloguj & & X & X \\
    \hline
    3 & Utwórz konto & & & X \\
    \hline
    4 & Wyświetl listę wszystkich użytkowników & & & X \\
    \hline
    5 & Usuń konto użytkownika & & & X \\
    \hline
    6 & Wyświetl dane swojego konta & & X & X \\
    \hline
    7 & Wyświetl dane dowolnego konta & & & X \\
    \hline
    8 & Wyświetl listę kontenerów przypisanych do swojego konta & & X & X \\
    \hline
    9 & Wyświetl listę wszystkich kontenerów & & & X \\
    \hline
    10 & Wyświetl status kontenera przypisanego do swojego konta & & X & X \\
    \hline
    11 & Wyświetl status kontenera przypisanego do dowolnego konta & & & X \\
    \hline
    12 & Uruchom kontener przypisany do swojego konta & & X & X \\
    \hline
    13 & Uruchom kontener przypisany do dowolnego konta & & & X \\
    \hline
    14 & Wyłącz kontener przypisany do swojego konta & & X & X \\
    \hline
    15 & Wyłącz kontener przypisany do dowolnego konta & & & X \\
    \hline
    16 & Modyfikuj dane personalne dowolnego konta & & & X \\
    \hline
    17 & Dodaj poziomy dostępów do dowolnego konta & & & X \\
    \hline
    18 & Usuń poziomy dostępów z dowolnego konta & & & X \\
    \hline
    19 & Modyfikuj dane personalne swojego konta & & X & X \\
    \hline
    20 & Zmień hasło do swojego konta & & X & X \\
    \hline
    21 & Zmień hasło do dowolnego konta & & & X \\
    \hline
    22 & Pobierz informacje adresowe o serwerze dla kontenerów LXC & & X & X \\
    \hline
    23 & Wyświetl listę utworzonych zadań zleconych ze swojego konta & & X & X \\
    \hline
    24 & Wyświetl listę utworzonych zadań zleconych z dowolnego konta & & & X \\
    \hline
    25 & Połącz z gniazdem z danymi o stanie wszystkich zadań & & & X \\
    \hline
    26 & Połącz z gniazdem z danymi stanie zadań dla swojego konta & & X & \\
    \hline
    27 & Rozłącz z gniazdem przesyłającym informacje o stanie zadań & & X & X \\
    \hline
  \end{tabular}
  \label{tab:przypadki_uzycia}
\end{table}

\section{Wzorce projektowe}
\label{sec:wzorce_projekctowe}

Podczas implementacji złożonych systemów informatycznych istotną rolę odgrywa struktura kodu
źródłowego oraz zdefiniowanie powiązań i sposobów komunikacji pomiędzy poszczególnymi elementami
wchodzącymi w skład tak zdefiniowanej struktury.
Wzorce projektowe są uniwersalnymi i sprawdzonymi metodami rozwiązywania tego typu problemów.
Poniżej zostały opisane najważniejsze ze wzorców, które zostały wykorzystanie podczas projektowania
pracy inżynierskiej.

\subsection{Model-Widok-Kontroler}
\label{subsec:webmvc}

Wzorzec Model-Widok-Kontroler, w skrócie MVC (ang.\ \textit{Model-View-Controller}) to wzorzec
architektoniczny służący do organizowania struktury aplikacji posiadających graficzne interfejsy
użytkownika.
Wzorzec MVC powstał oryginalnie dla aplikacji desktopowych, dlatego w projekcie została
wykorzystana jego zmodyfikowana wersja dla aplikacji internetowych.
Wzorzec MVC dla aplikacji internetowych definiuje strukturę aplikacji podzieloną na trzy części:

\begin{itemize}
  \item Model.
  Reprezentuje logikę biznesową aplikacji.
  \item Widok.
  Jest odpowiedzialny na wyświetlenie graficznego interfejsu użytkownika.
  Wysyła żądania generowane przez użytkownika do kontrolera oraz aktualizuje widoki na podstawie
  odpowiedzi kontrolera na te żądania.
  \item Kontroler.
  Koordynuje komunikację pomiędzy widokiem a modelem.
  Przekazuje żądania wysłane z widoku do modelu oraz przekazuje odpowiedzi na te żądania do widoku
  na podstawie komunikatów z modelu.
\end{itemize}

\subsection{Wstrzykiwanie zależności}
\label{subsec:di}

Wstrzykiwanie zależności (ang.\ \textit{Dependency Injection}) w programowaniu obiektowym
polega na dostarczaniu zależności dla obiektu poprzez inny obiekt.
Dzięki zastosowaniu wzorca wstrzykiwania obiekty są niezależne od sposobów tworzenia
pozostałych obiektów, powiązania pomiędzy obiektami stają się bardziej luźne.
Umożliwia to m.in.\ realizację paradygmatu odwrócenia sterowania
(ang.\ \textit{Inversion of Control})oraz ułatwia testowanie aplikacji dzięki możliwości
wstrzykiwania atrapy obiektów.

\section{Warstwa danych}
\label{sec:warstwa_danych}

W skład warstwy danych projektu pracy inżynierskiej wchodzą dwa systemy informatyczne.
Pierwszy z nich to system zarządzania relacyjnymi bazami danych, w której są składowane dane
użytkowników systemu i przypisanych do nich kontenerów.
Drugi system informatyczny warstwy danych to serwer odpowiedzialny za składowanie kontenerów LXC\@.
W poniższych podsekcjach znajduje się szczegółowy opis każdego z tych modułów.

\section{Serwer składujący kontenery LXC}
\label{sec:serwer_lxc}

W skład warstwy danych projektu pracy inżynierskiej wchodzą dwa systemy informatyczne.
Pierwszy z nich to serwer odpowiedzialny za składowanie kontenerów LXC\@.
Kontenery LXC stanowią część systemu operacyjnego. Z tego powody nie mogą być w prosty sposób
przechowywane w bazach danych, które specjalizują się w przechowywaniu innych typów danych.
Aby rozwiązać ten problem w projekcie pracy inżynierskiej, podjęto decyzje o wykorzystaniu
osobnego serwera, którego zadaniem byłoby składowanie kontenerów.


\clearpage