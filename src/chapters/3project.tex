\chapter{Projekt systemu}
\label{ch:projekt_systemu}

System informatyczny będący tematem pracy inżynierskiej został zrealizowany w architekturze
wielowarstwowej.
W dalszych sekcjach znajduje się szczegółowy opis każdej z warstw.
Przedstawione zostały poziomy dostępu użytkowników, modele danych,
wykorzystane wzorce projektowe, zabezpieczenia aplikacji oraz sposoby przepływu informacji
w przekroju całego systemu informatycznego.
Ogólny schemat przepływu danych pomiędzy elementami systemu został przedstawiony
na rysunku~\ref{fig:arch}.

\begin{figure}[H]
  \centering
  \includegraphics[scale=0.59]{figures/arch.png}
  \caption{Schemat przepływu danych pomiędzy elementami systemu}
  \label{fig:arch}
\end{figure}
\newpage

\section{Poziomy dostępów}
\label{sec:przypadki_uzycia}

W zaprojektowanym systemie istnieje kilka poziomów dostępów dla użytkownika.
Poszczególne poziomy dostępów to:
\begin{itemize}
  \item Poziom Gościa, jest przypisywany dla niezalogowanych użytkowników.
  \item Poziom Klienta, jest przypisywany do zalogowanych użytkowników, których konto ma przypisany
  poziom dostępu klienta.
  Umożliwia dostęp do danych związanych z danym kontem i ich ograniczoną modyfikację.
  \item Poziom Administratora, jest przypisywany do zalogowanych użytkowników, których konto ma
  przypisany poziom dostępu administratora.
  Umożliwia dostęp do danych o użytkownikach i kontenerach oraz modyfikację tych danych.
\end{itemize}

Wszystkie przypadki użycie dla poszczególnych poziomów dostępów przedstawia
tabela~\ref{tab:przypadki_uzycia}.
Litery G, K, A oznaczają odpowiednio poziomy dostępów Gościa, Klienta i Administratora.

\newpage
\begin{table}[H]
  \centering
  \caption{Macierz decyzyjna przypadków użycia}
  \begin{tabular}{|r|l|c|c|c|}
    \hline
    Lp. & Przypadek użycia & G & K & A \\
    \hline
    1 & Zaloguj & X & &  \\
    \hline
    2 & Wyloguj & & X & X \\
    \hline
    3 & Utwórz konto & & & X \\
    \hline
    4 & Wyświetl listę wszystkich użytkowników & & & X \\
    \hline
    5 & Usuń konto użytkownika & & & X \\
    \hline
    6 & Wyświetl dane swojego konta & & X & X \\
    \hline
    7 & Wyświetl dane dowolnego konta & & & X \\
    \hline
    8 & Wyświetl listę kontenerów przypisanych do swojego konta & & X & X \\
    \hline
    9 & Wyświetl listę wszystkich kontenerów & & & X \\
    \hline
    10 & Wyświetl status kontenera przypisanego do swojego konta & & X & X \\
    \hline
    11 & Wyświetl status kontenera przypisanego do dowolnego konta & & & X \\
    \hline
    12 & Uruchom kontener przypisany do swojego konta & & X & X \\
    \hline
    13 & Uruchom kontener przypisany do dowolnego konta & & & X \\
    \hline
    14 & Wyłącz kontener przypisany do swojego konta & & X & X \\
    \hline
    15 & Wyłącz kontener przypisany do dowolnego konta & & & X \\
    \hline
    16 & Modyfikuj dane personalne dowolnego konta & & & X \\
    \hline
    17 & Dodaj poziomy dostępów do dowolnego konta & & & X \\
    \hline
    18 & Usuń poziomy dostępów z dowolnego konta & & & X \\
    \hline
    19 & Modyfikuj dane personalne swojego konta & & X & X \\
    \hline
    20 & Zmień hasło do swojego konta & & X & X \\
    \hline
    21 & Zmień hasło do dowolnego konta & & & X \\
    \hline
    22 & Pobierz informacje adresowe o serwerze dla kontenerów LXC & & X & X \\
    \hline
    23 & Wyświetl listę utworzonych zadań zleconych ze swojego konta & & X & X \\
    \hline
    24 & Wyświetl listę utworzonych zadań zleconych z dowolnego konta & & & X \\
    \hline
    25 & Połącz z gniazdem z danymi o stanie wszystkich zadań & & & X \\
    \hline
    26 & Połącz z gniazdem z danymi stanie zadań dla swojego konta & & X & \\
    \hline
    27 & Rozłącz z gniazdem przesyłającym informacje o stanie zadań & & X & X \\
    \hline
  \end{tabular}
  \label{tab:przypadki_uzycia}
\end{table}

\section{Wzorce projektowe}
\label{sec:wzorce_projekctowe}

Podczas implementacji złożonych systemów informatycznych istotną rolę odgrywa struktura kodu
źródłowego oraz zdefiniowanie powiązań i sposobów komunikacji pomiędzy poszczególnymi elementami
wchodzącymi w skład tak zdefiniowanej struktury.
Wzorce projektowe są uniwersalnymi i sprawdzonymi metodami rozwiązywania tego typu problemów.
Poniżej zostały opisane najważniejsze ze wzorców, które zostały wykorzystanie podczas projektowania
pracy inżynierskiej.

\subsection{Model-Widok-Kontroler}
\label{subsec:webmvc}

Wzorzec Model-Widok-Kontroler, w skrócie MVC (ang.\ \textit{Model-View-Controller}) to wzorzec
architektoniczny służący do organizowania struktury aplikacji posiadających graficzne interfejsy
użytkownika.
Wzorzec MVC powstał oryginalnie dla aplikacji desktopowych, dlatego w projekcie została
wykorzystana jego zmodyfikowana wersja dla aplikacji internetowych.

Wzorzec MVC dla aplikacji internetowych definiuje strukturę aplikacji podzieloną na trzy części:

\begin{itemize}
  \item \textbf{Model}.
  Reprezentuje logikę biznesową aplikacji.
  \item \textbf{Widok}.
  Jest odpowiedzialny na wyświetlenie graficznego interfejsu użytkownika.
  Wysyła żądania generowane przez użytkownika do kontrolera oraz aktualizuje widoki na podstawie
  odpowiedzi kontrolera na te żądania.
  \item \textbf{Kontroler}.
  Koordynuje komunikację pomiędzy widokiem a modelem.
  Przekazuje żądania wysłane z widoku do modelu oraz przekazuje odpowiedzi na te żądania do widoku
  na podstawie komunikatów z modelu.
\end{itemize}

\subsection{Wstrzykiwanie zależności}
\label{subsec:di}

Wstrzykiwanie zależności (ang.\ \textit{Dependency Injection}) w programowaniu obiektowym
polega na dostarczaniu zależności dla obiektu poprzez inny obiekt.
Dzięki zastosowaniu wzorca wstrzykiwania obiekty są niezależne od sposobów tworzenia
pozostałych obiektów, powiązania pomiędzy obiektami stają się bardziej luźne.
Umożliwia to m.in.\ realizację paradygmatu odwrócenia sterowania
(ang.\ \textit{Inversion of Control})oraz ułatwia testowanie aplikacji dzięki możliwości
wstrzykiwania atrapy obiektów.

\section{Serwer LXC}
\label{sec:serwer_lxc}

W skład warstwy danych projektu pracy inżynierskiej wchodzą dwa systemy informatyczne.
Pierwszy z nich to serwer odpowiedzialny za składowanie kontenerów LXC\@.
Kontenery LXC stanowią część systemu operacyjnego.
Z tego powodu nie mogą być w prosty sposób przechowywane w bazach danych, które specjalizują
się w przechowywaniu innych typów danych.
Aby rozwiązać ten problem w projekcie pracy inżynierskiej, podjęto decyzje o wykorzystaniu
osobnego serwera, którego zadaniem byłoby składowanie kontenerów.
W poniższych podsekcjach opisano technologię kontenerów Linux oraz wymaganą konfigurację serwera,
która umożliwia jego integrację z pozostałymi modułami projektu.

\subsection{Kontenery Linux}
\label{subsec:kontenery_linux}

Konteneryzacja jest kolejnym logicznym krokiem w rozwoju technologii wirtualizacji.
Kontenery dostarczają mechanizm wirtualizacji na poziomie systemu operacyjnego jak również
na poziomie aplikacji.
Projekt pracy inżynierskiej wykorzystuje technologię konteneryzacji na poziomie systemu
operacyjnego opartego o jądro Linux.

Główne zalety kontenerów Linux to:

\begin{itemize}
  \item Dostarczają kompletne środowisko systemu operacyjnego, które jest izolowane.
  \item Umożliwiają opakowanie i izolację aplikacji wraz z ich całym środowiskiem uruchomieniowym.
  \item Dostarczają przenośne i lekkie środowisko aplikacyjne.
  \item Pomagają w lepszym wykorzystaniu zasobów w centrach danych.
  \item Ułatwiają tworzenie środowisk testowych i produkcyjnych podczas implementacji
  systemów informatycznych.
\end{itemize}

Kontener może być zdefiniowany jako osobny obraz systemu operacyjnego, zawierający izolowany
zestaw aplikacji oraz ich zależności, dzięki któremu mogą one być odseparowane od
maszyny gospodarza.
Może istnieć wiele kontenerów działających jednocześnie w obrębie jednej maszyny gospodarza.

Kontenery dzielą się na dwa typy:

\begin{itemize}
  \item \textbf{Kontenery na poziomie systemu operacyjnego}.
  Cały system operacyjny działa w izolowanym środowisku na maszynie gospodarza, współdzieląc
  to samo jądro systemu.
  \item \textbf{Kontenery na poziomie aplikacyjnym}.
  Aplikacja lub serwis i minimalny zestaw procesów wymaganych przez tą aplikacje, działa
  w izolowanym środowisku na maszynie gospodarza.
\end{itemize}

Konteneryzacja różni się od tradycyjnej technologii wirtualizacji i oferuje wiele dodatkowych
zalet w porównaniu do tradycyjnej wirtualizacji:

\begin{itemize}
  \item Kontenery są znacznie lżejsze od tradycyjnych maszyn wirtualnych.
  \item Wirtualnie maszyny wymagają warstw emulacji, zarówno sprzętu jak i oprogramowania,
  które pochłaniają więcej zasobów i spowalniają działanie.
  Kontenery nie wymagają warstw emulacji.
  \item Kontenery współdzielą zasoby z maszyną gospodarza, izolując przestrzeń użytkownika
  (ang.\ \textit{user space}) i procesy.
  \item Dzięki lekkości kontenerów, na maszynie gospodarza może ich działać znacznie więcej
  niż tradycyjnych maszyn wirtualnych.
  \item Uruchomienie kontenera jest niemal natychmiastowe, maszyny wirtualne startują
  znacznie wolniej.
\end{itemize}

Poniżej na rysunkach~\ref{fig:vms} i~\ref{fig:lxcs} przedstawiono sposób w jaki tradycyjne
maszyny wirtualne i kontenery Linux są zorganizowanie w systemie operacyjnym gospodarza.

\begin{figure}[H]
  \centering
  \includegraphics[scale=0.81]{figures/vms.png}
  \caption{Maszyny wirtualne}
  \label{fig:vms}
\end{figure}

\begin{figure}[H]
  \centering
  \includegraphics[scale=0.81]{figures/lxcs.png}
  \caption{Kontenery Linux}
  \label{fig:lxcs}
\end{figure}

W systemie Linux kontenery mogą działać dzięki specyficznym funkcjonalnościom jądra Linux.
Najważniejsze z tych funkcjonaliści to:

\begin{itemize}
  \item \textbf{Grupy kontroli (\textit{cgroups})}.
  Dostarczają mechanizmy grupowania zadań i procesów w hierarchiczne grupy.
  \item \textbf{Przestrzenie nazw (\textit{namespaces})}.
  Dostarczają warstwy abstrakcji dla globalnych zasobów systemu, które będą widoczne
  dla procesów wewnątrz tej samej przestrzeni nazw jako izolowania instancja globalnego
  zasobu.
  \item \textbf{System plików lub \textit{rootfs}}.
  Obraz kontenera posiada własny system plików o własnym rdzeniu o podobnej strukturze do
  systemu plików zamontowanego na każdej maszynie z systemem operacyjnym GNU/Linux.
\end{itemize}

\section{RDBMS}
\label{sec:rdbms}

Poza kontenerami, projekt pracy inżynierskiej przechowuje również informacje o kontach użytkowników,
do jakich kontenerów posiadają one dostęp oraz podstawowe dane o tych kontenerach
jak np.\ adres IP\@.
Poniżej na rysunku~\ref{fig:datarel} przedstawiono relacje pomiędzy tymi danymi.

\begin{figure}[H]
  \centering
  \includegraphics[scale=0.75]{figures/datarel.png}
  \caption{Relacje między strukturami danych}
  \label{fig:datarel}
\end{figure}

Z tego powodu w skład warstwy danych projektu pracy inżynierskiej wchodzi również drugi system.
Jest to system zarządzania relacyjną bazą danych, inaczej
RDBMS (ang.\ \textit{Relational Database Management System}).
RDBMS dostarcza narzędzia potrzebne do interakcji z bazami danych, np. poprzez dostarczoną
implementację specyfikacji języka SQL\@.
Dodatkowo zawiera mechanizmy pozwalające zachować integralność danych. Najważniejszym
takim mechanizmem są transakcje bazodanowe, które opisano bardziej szczegółowo w następnej
podsekcji.

\subsection{Transakcje bazodanowe}
\label{subsec:transakcje}

Konteneryzacja jest kolejnym logicznym krokiem w rozwoju technologii wirtualizacji.
Kontenery dostarczają mechanizm wirtualizacji na poziomie systemu operacyjnego jak również
na poziomie aplikacji.
Projekt pracy inżynierskiej wykorzystuje technologię konteneryzacji na poziomie systemu
operacyjnego opartego o jądro Linux.

Główne zalety kontenerów Linux to:

\clearpage