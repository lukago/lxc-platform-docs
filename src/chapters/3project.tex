\chapter{Projekt systemu}
\label{ch:projekt_systemu}

System informatyczny będący tematem pracy inżynierskiej został zrealizowany w architekturze
wielowarstwowej.
W dalszych sekcjach znajduje się szczegółowy opis każdej z warstw.
Przedstawione zostały poziomy dostępu użytkowników, modele danych,
wykorzystane wzorce projektowe, zabezpieczenia aplikacji oraz sposoby przepływu informacji
w przekroju całego systemu informatycznego.
Ogólny schemat przepływu danych pomiędzy elementami systemu został przedstawiony
na rysunku~\ref{fig:arch}.

\begin{figure}[H]
  \centering
  \includegraphics[scale=0.59]{figures/arch.png}
  \caption{Schemat przepływu danych pomiędzy elementami systemu}
  \label{fig:arch}
\end{figure}
\newpage

\section{Poziomy dostępów}
\label{sec:przypadki_uzycia}

W zaprojektowanym systemie istnieje kilka poziomów dostępów dla użytkownika.
Poszczególne poziomy dostępów to:
\begin{itemize}
  \item Poziom Gościa, jest przypisywany dla niezalogowanych użytkowników.
  \item Poziom Klienta, jest przypisywany do zalogowanych użytkowników, których konto ma przypisany
  poziom dostępu klienta.
  Umożliwia dostęp do danych związanych z danym kontem i ich ograniczoną modyfikację.
  \item Poziom Administratora, jest przypisywany do zalogowanych użytkowników, których konto ma
  przypisany poziom dostępu administratora.
  Umożliwia dostęp do danych o użytkownikach i kontenerach oraz modyfikację tych danych.
\end{itemize}

Wszystkie przypadki użycie dla poszczególnych poziomów dostępów przedstawia
tabela~\ref{tab:przypadki_uzycia}.
Litery G, K, A oznaczają odpowiednio poziomy dostępów Gościa, Klienta i Administratora.

\newpage
\begin{table}[H]
  \centering
  \caption{Macierz decyzyjna przypadków użycia}
  \begin{tabular}{|r|l|c|c|c|}
    \hline
    Lp. & Przypadek użycia & G & K & A \\
    \hline
    1 & Zaloguj & X & &  \\
    \hline
    2 & Wyloguj & & X & X \\
    \hline
    3 & Utwórz konto & & & X \\
    \hline
    4 & Wyświetl listę wszystkich użytkowników & & & X \\
    \hline
    5 & Usuń konto użytkownika & & & X \\
    \hline
    6 & Wyświetl dane swojego konta & & X & X \\
    \hline
    7 & Wyświetl dane dowolnego konta & & & X \\
    \hline
    8 & Wyświetl listę kontenerów przypisanych do swojego konta & & X & X \\
    \hline
    9 & Wyświetl listę wszystkich kontenerów & & & X \\
    \hline
    10 & Wyświetl status kontenera przypisanego do swojego konta & & X & X \\
    \hline
    11 & Wyświetl status kontenera przypisanego do dowolnego konta & & & X \\
    \hline
    12 & Uruchom kontener przypisany do swojego konta & & X & X \\
    \hline
    13 & Uruchom kontener przypisany do dowolnego konta & & & X \\
    \hline
    14 & Wyłącz kontener przypisany do swojego konta & & X & X \\
    \hline
    15 & Wyłącz kontener przypisany do dowolnego konta & & & X \\
    \hline
    16 & Modyfikuj dane personalne dowolnego konta & & & X \\
    \hline
    17 & Dodaj poziomy dostępów do dowolnego konta & & & X \\
    \hline
    18 & Usuń poziomy dostępów z dowolnego konta & & & X \\
    \hline
    19 & Modyfikuj dane personalne swojego konta & & X & X \\
    \hline
    20 & Zmień hasło do swojego konta & & X & X \\
    \hline
    21 & Zmień hasło do dowolnego konta & & & X \\
    \hline
    22 & Pobierz informacje adresowe o serwerze dla kontenerów LXC & & X & X \\
    \hline
    23 & Wyświetl listę utworzonych zadań zleconych ze swojego konta & & X & X \\
    \hline
    24 & Wyświetl listę utworzonych zadań zleconych z dowolnego konta & & & X \\
    \hline
    25 & Połącz z gniazdem z danymi o stanie wszystkich zadań & & & X \\
    \hline
    26 & Połącz z gniazdem z danymi stanie zadań dla swojego konta & & X & \\
    \hline
    27 & Rozłącz z gniazdem przesyłającym informacje o stanie zadań & & X & X \\
    \hline
  \end{tabular}
  \label{tab:przypadki_uzycia}
\end{table}