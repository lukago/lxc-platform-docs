\chapter{Podsumowanie}
\label{ch:podsumowanie}

Projekt pracy inżynierskiej miał na celu stworzenie platformy dla studentów, za pomocą której
będą oni mogli korzystać z wielu systemów operacyjnych opartych o jądro Linux.
Cel pracy opierał się o wykorzystanie do budowy tej platformy kontenerów Linux
oraz sprawdzenie czy ta technologia pozwoli stworzyć taką platformę.

Kontenery Linux spełniły swoje zadanie w zrealizowanym systemie.
Ich wykorzystanie umożliwiło stworzenie wielu izolowanych przestrzeni odzwierciedlających system
operacyjny Linux, do których użytkownik systemu może się połączyć w prosty sposób za pomocą
protokołu SSH\@.
Jednocześnie użycie technologii LXC pozwoliło wykorzystać szereg zalet konteneryzacji.
Jest to przede wszystkim szybkość i oszczędność zasobów, dzięki współdzieleniu jądra systemu
gospodarza.

Doskonale sprawdziło się również narzędzie LXD, które pełniło rolę oprogramowania zarządzającego
kontenerami na serwerze składującym LXC\@.
Dostarczyło ono prosty interfejs dostępny z poziomu powłoki, który umożliwił tworzenie kontenerów,
startowanie i zatrzymywanie ich, pobieranie informacji o ich stanie,
czy przekazywanie portów kontenera na porty znajdujące się w systemie gospodarza.

Podczas implementacji, kontenery Linux potwierdziły swoją stabilność.
Raz uruchomione kontenery, pozostawały w takim stanie.
Nie napotkano na nieprzewidziane wyłączenie kontenera spowodowane błędem.
Kontenery uruchamiały się szybko, bez problemu były dostępne z poziomu sieci internetowej.
Zgodnie z założeniami technologii LXC, dane kontenerów były poprawnie przechowywane i dostępne po
ich ponownym uruchomieniu.

Zrealizowany projekt jest złożonym systemem informatycznym, dostępnym dla wielu użytkowników
za pośrednictwem przeglądarki internetowej.
Z tego powodu, poza kontenerami Linux, wykorzystano również szereg innych technologii,
zadbano o bezpieczeństwo i inne dobre praktyki inżynierii oprogramowania.
System składuje dane użytkowników i ich powiązania z kontenerami w relacyjnej bazie danych
w bezpieczny sposób, hasła użytkowników są przechowywane w formie niejawnej.
Aplikacja serwerowa i kliencka są od siebie odseparowane i komunikują się
za pomocą interfejsów HTTP, opartych o architekturę REST\@.
Aplikacja serwerowa jest bezstanowa, co umożliwi jej prostą skalowalność w przyszłości.
Działania na kontenerach są wykonywane asynchronicznie, a wynik tych działań wysyłany jest
za pośrednictwem technologii WebSocket, co pozwoliło oszczędzić zasoby sieci oraz aplikacji
serwerowerj i klienckiej.
Aplikacja klienta została wykonana jako jednostronicowa aplikacja internetowa, dzięki czemu
nie tylko uzyskano przyjazny dla użytkownika interfejs, ale również pozwoliło to oszczędzić
zasoby na przekroju całego systemu informatycznego, dzięki ograniczeniu ilości wysyłanych danych.
Zwiększono również bezpieczeństwo aplikacji, wzięto pod uwagę ataki CSRF, XSS czy wstrzykiwania
kodu oraz wymuszono wykorzystanie protokołu HTTPS\@.

Cel pracy inżynierskiej został osiągnięty.
Zrealizowany system informatyczny udowadnia, że wykorzystanie kontenerów Linux pozwoliło stworzyć
platformę dla studentów umożliwiającą korzystanie z wielu systemów GNU/Linux.
Projekt realizuje również dodatkowe wymaganie funkcjonalne i niefunkcjonalne związane
z bezpieczeństwem, czy sposobami integracji pomiędzy warstwami systemu.
System spełnia potrzeby wykazane podczas analizy istniejących rozwiązań.

\clearpage
