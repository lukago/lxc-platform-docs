\chapter{Wstęp}
\label{ch:wstep}
Obecnie wszelkiego rodzaju rozwiązania wykorzystujące chmury obliczeniowe bardzo szybko
zysują na popularności.
Wiele przedsiębiorstw decycuje się na użycie wspomnianego rozwiązania, ponieważ jest bardziej
korzystne pod wzlędem finansowym oraz dostarcza wiele sprawdzonych i gotowych rozwiązań.
W przypadku wykorzystania chmury obliczeniowej nie ma potrzebu zakupu własnego sprzętu
komuputerowego oraz zatrudniania specjalistów do stworzenia i utrzymania
infrastruktury informatycznej.

W Polsce w 2018 r.\ z płatnych usług chmury obliczeniowej korzystało 11,5\% przedsiębiorstw.
Jest to o 1.5p.\ proc.\ więcej niż w roku 2017, co potwierdza stały i szybki wzrost popularności
w tej dziedzine~\cite{einstein}.
Podobny trend utrzymuję się nie tylko w Polsce ale i na całym świecie.
W 2018 r.\ 26\% przedsiębiorstw w Unii Europejskiej korzystało z chmur
obliczeniowych~\cite{einstein}.

Jednym z rozwiązań chmury obliczeniowej jest model PaaS (Platform as a Service, z angielskiego
"platforma jako usługa"), w któchych zewnętrzny system zarząca całym środowiskiem aplikacyjnym
i udostępnia wybrane usługi użytkownikom.
Aby ułatwić zarządzanie wieloma systemami operacyjnymi opartymi o jądro Linux,
które były wykorzystywanie przez studentów Politechniki Łódzkiej na przedmiocie Systemy Operacyjne,
prowadzący przedmiot wyrazili zapotrzebowanie na tego typu system informatyczny.

\section{Cel i zakres pracy}
\label{sec:cel i zakres pracy}

Celem tej pracy jest zbadanie możliwości jakie daje zastosowanie kontenerów
Linux w zakresie tworzenia infrastruktury PaaS na przykładzie systemu do zastosowań dydaktycznych.
System ten będzie integrował się z kontenerami Linux, inaczej LXC (ang.\ Linux Containers) za pomocą
managera kontenerów systemowóych LXD\@.
Kontenery LXC będą składowane na dedykowanym serwerze, może to być inny serwer od tego na którym
działa aplikacja, ale nie musi.
Komunikacja ze zdalnym serwerem składującym kontenery będzie odbywać się z wykorzystaniem
protokołu SSH\@.
Dane o użytkownikach i informacje o składowanych kontenerach będą składowanie w relacyjnej
bazie danych.
System będzie składaj się z aplikacji serwerowej oraz aplikacji klienckiej.


Do zakresu pracy inżynierskiej należą następujące czynności:
\begin{enumerate}
  \item Przygotowanie osobnych repozytoriów kodu żródłowego,
  przygotowanie środowiska programistycznego i narzędzi deweloperskich
  potrzebnych do stworzenia aplikacji.
  \item Utworzenie projektu aplikacji serwerowej,
  zarządanego za pomocą narzędzia narzędzia automatyzującego budowę oprogramowania Apache Maven
  \item Utworzenie projektu aplikacji klienckiej,
  zarządanego za pomocą narzędzia narzędzia automatyzującego budowę oprogramowania npm.
  \item Skonfigurowanie serwera do zarządzania kontenerami Linux.
  \item Skonfigurowanie systemu zarządzania bazami danych PostgreSQL\@.
  \item Określenie poziomów dostępu dla użytkowników systemie oraz zaprojektowanie
  przypadków użycia.
  \item Zaprojektowanie struktur danych reprezentujących zasoby zarządane przez system.
  \item Zaprojektowanie interfejstu programistycznego aplikacji serwerowej
  w oparciu o REST i gniazda.
  \item Zaprojektowanie warstwy logiki biznesowej systemu oraz jej implementacja.
  \item Konfiguracja połączenia między serwerem z aplikacją serwerową
  a serwerem do zarządzania kontenerami Linux.
  \item Zaprojektowanie oraz implementacja aplikacji klienckiej.
  \item Internacjonalizacja aplikacji klienckiej oraz jej lokalizacja w języku polskim i angielskim.
  \item Zabezpieczenie systemu
  \item Konteneryzacja aplikacji klienckiej i serwerowej z wykorzystaniem narzędzia Docker,
  z możliwosicią zmainy konfiguracji aplikacji bez potrzeby jej ponownego kompilowania.
  \item Przygotowanie danych inicjujących dla utworzonej aplikacji.
\end{enumerate}

\section{Dostępne rozwiązania}
\label{sec:dostepne rozwiazania}

Platformy chmurowe w modelu Paas są obecne na rynku od kilkunastu lat.
Obecnie są to już bardzo dojrzałe aplikacje posiadające
bardzo szeroką gamę usług i funcjonalności.
Najpopularniejsze z nich to:

\begin{itemize}
  \item OpenShift,
  \item Heroku,
  \item Amazon Web Services.
\end{itemize}

Wymienione rozwiązania posiadają bardzo podobne możliwości.
Większość z nich wykorzystuje zarówno kontenery Docker jak i LXC\@.
Użytkownicy mogą korzystać z platformy OpenShift przez konsolę Web, linię komend lub poprzez
zintegrowane środowisko.
Posiadają szereg udogodnień dla konfiguracji szybkich wdrożeń aplikacji oraz dla szerokiego zakresu
aplikacji rozproszonych.
Korzystanie z serwisów wiąże się z opłatami adekwatnymi do zużycia zasobów infrastruktury
informatyczniej na którch jest wrożona aplikacja.

Z analizy wynika powyższych rozwiązań, że na rynku istnieją wspierane rozwiązania PaaS
jednak wszystkie rozwiązania wiążą się z opłatami za korzystanie z usługi.
Dodatkowo każde z rozwiązań skupia się przede wszystkim na usprawnianiu
procesu wdrożenia oprogramowania.
Z tych powodów potrzebna jest implementacja własnego systemu wspomagającego zarządzanie
wieloma kontenerami LXC dostosowanego do potrzeb dydaktycznych przedmiotu Systemy Operacyjne.

\section{Założenia}
\label{sec:zalozenia}

Zaprojektowany system powinien spełniać następujące założenia:

\begin{itemize}
  \item System jest wielodostępny - może korzystać z niego wielu użytkowników jednocześnie.
  \item Konto każdego użytkownika systemu ma przypisaną jedną lub więcej ról, z którymi wiążą się
  określone poziomy dostępu.
  Możliwe role to klient lub administrator.
  \item System jest wielodostępny - może korzystać z niego wielu użytkowników jednocześnie.
  \item System jest wielodostępny - może korzystać z niego wielu użytkowników jednocześnie.
  \item System jest wielodostępny - może korzystać z niego wielu użytkowników jednocześnie.
  \item System jest wielodostępny - może korzystać z niego wielu użytkowników jednocześnie.
  \item System jest wielodostępny - może korzystać z niego wielu użytkowników jednocześnie.
  \item System jest wielodostępny - może korzystać z niego wielu użytkowników jednocześnie.
  \item System jest wielodostępny - może korzystać z niego wielu użytkowników jednocześnie.
\end{itemize}

\section{Stos technologiczny}
\label{sec:stos technologiczny}

W ramach pracy inżynierskiej, do zbudowania systemu informatycznego zostało użytych wiele
technologii.
Ich spis wraz z wersjami w jakich zostały użyte oraz licencjami znajduje się
w tabeli~\ref{tab:technologie}.

\begin{table}[H]
  \centering
  \begin{tabular}{|c|c|c|}
    \hline
    Technologia & Wersja & Licencja \\
    \hline
    OpenJDK & 11 & GPL \\
    \hline
    OpenJDK & 11 & GPL \\
    \hline
    OpenJDK & 11 & GPL \\
    \hline
    OpenJDK & 11 & GPL \\
    \hline
    OpenJDK & 11 & GPL \\
    \hline
    OpenJDK & 11 & GPL \\
    \hline
    OpenJDK & 11 & GPL \\
    \hline
    OpenJDK & 11 & GPL \\
    \hline
    OpenJDK & 11 & GPL \\
    \hline
    OpenJDK & 11 & GPL \\
    \hline
  \end{tabular}
  \caption{Technologie wykorzystane w ramach projektu}
  \label{tab:technologie}
\end{table}

\clearpage