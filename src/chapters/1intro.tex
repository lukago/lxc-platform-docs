\chapter{Wstęp}
\label{ch:wstęp}
Obecnie wszelkiego rodzaju rozwiązania wykorzystujące chmury obliczeniowe bardzo szybko
zyskują na popularności.
Wiele przedsiębiorstw decyduje się na użycie wspomnianego rozwiązania, ponieważ jest bardziej
korzystne pod względem finansowym oraz dostarcza wiele sprawdzonych i gotowych rozwiązań.
W przypadku wykorzystania chmury obliczeniowej nie ma potrzeby zakupu własnego sprzętu
komputerowego oraz zatrudniania specjalistów do stworzenia i utrzymania
infrastruktury informatycznej.

W Polsce w 2018 r.\ z płatnych usług chmury obliczeniowej korzystało 11,5\% przedsiębiorstw.
Jest to o 1.5p.\ proc.\ więcej niż w roku 2017, co potwierdza stały i szybki wzrost popularności
w tej dziedzine~\cite{gus:cloud}.
Podobny trend utrzymuje się nie tylko w Polsce, ale również w pozostałych krajach.
W 2018 r.\ 26\% przedsiębiorstw w Unii Europejskiej korzystało z chmur
obliczeniowych~\cite{eurostat:cloud}.

Jednym z rozwiązań chmury obliczeniowej jest model PaaS (ang. \textit{Platform as a Service})
tj\. platforma jako usługa, w której zewnętrzny system zarządza całym środowiskiem aplikacyjnym
i udostępnia wybrane usługi użytkownikom.
Aby ułatwić zarządzanie wieloma systemami operacyjnymi opartymi o jądro Linux,
które były wykorzystywanie przez studentów Politechniki Łódzkiej na przedmiocie Systemy Operacyjne,
prowadzący przedmiot wyrazili zapotrzebowanie na tego typu system informatyczny.

\section{Cel i zakres pracy}
\label{sec:cel i zakres pracy}

Celem tej pracy jest zbadanie możliwości, jakie daje zastosowanie kontenerów
Linux w zakresie tworzenia infrastruktury PaaS na przykładzie systemu do zastosowań dydaktycznych.
System ten będzie integrował się z kontenerami Linux, inaczej LXC (ang.\ \textit{Linux Containers}),
za pomocą managera kontenerów systemowych LXD\@.
Kontenery LXC będą składowane na dedykowanym serwerze, może to być inny serwer od tego na którym
działa aplikacja, ale nie musi.
Komunikacja ze zdalnym serwerem składującym kontenery będzie odbywać się z wykorzystaniem
protokołu SSH\@.
Dane o użytkownikach i informacje o składowanych kontenerach będą zapisywane w relacyjnej
bazie danych.
System będzie składał się z aplikacji serwerowej oraz aplikacji klienckiej.
System umożliwi administratorowi zarządzanie wieloma kontenerami LXC oraz udostępni każdemu z
użytkowników bezpieczny dostęp do własnego kontenera, który będzie zachowywał swój stan.

Do zakresu pracy inżynierskiej należą następujące czynności:
\begin{enumerate}
  \item Przygotowanie osobnych repozytoriów kodu źródłowego dla aplikacji serwerowej i klienckiej,
  przygotowanie środowiska programistycznego i narzędzi deweloperskich
  potrzebnych do stworzenia aplikacji.
  \item Utworzenie projektu aplikacji serwerowej,
  zarządzanego za pomocą narzędzia narzędzia automatyzującego budowę oprogramowania Apache Maven.
  \item Utworzenie projektu aplikacji klienckiej,
  zarządzanego za pomocą narzędzia narzędzia automatyzującego budowę oprogramowania npm.
  \item Skonfigurowanie serwera do zarządzania kontenerami Linux.
  \item Skonfigurowanie systemu zarządzania bazami danych PostgreSQL\@.
  \item Określenie poziomów dostępu dla użytkowników systemie oraz zaprojektowanie
  przypadków użycia.
  \item Zaprojektowanie struktur danych reprezentujących zasoby zarządzane przez system.
  \item Zaprojektowanie interfejsu programistycznego aplikacji serwerowej w oparciu
  o REST i gniazda.
  \item Zaprojektowanie warstwy logiki biznesowej systemu oraz jej implementacja.
  \item Konfiguracja połączenia między serwerem z aplikacją serwerową a serwerem do zarządzania
  kontenerami Linux.
  \item Zaprojektowanie oraz implementacja aplikacji klienckiej.
  \item Internacjonalizacja aplikacji klienckiej oraz jej lokalizacja w języku polskim i angielskim.
  \item Zabezpieczenie systemu.
  \item Konteneryzacja aplikacji klienckiej i serwerowej z wykorzystaniem narzędzia Docker,
  z możliwością zmiany konfiguracji aplikacji bez potrzeby jej ponownego kompilowania.
  \item Przygotowanie danych inicjujących dla utworzonej aplikacji.
\end{enumerate}

\section{Dostępne rozwiązania}
\label{sec:dostępne rozwiązania}

Platformy chmurowe w modelu PaaS są obecne na rynku od kilkunastu lat.
Obecnie są to już bardzo dojrzałe aplikacje, posiadające bardzo szeroką gamę usług
i funkcjonalności.
Najpopularniejsze z nich to:

\begin{itemize}
  \item OpenShift,
  \item Heroku,
  \item Amazon Web Services.
\end{itemize}

Wymienione rozwiązania posiadają bardzo podobne możliwości.
Większość z nich wykorzystuje zarówno kontenery Docker jak i LXC\@.
Użytkownicy mogą korzystać z platformy OpenShift przez konsolę Web, linię komend lub poprzez
zintegrowane środowisko.
Posiadają szereg udogodnień dla konfiguracji szybkich wdrożeń aplikacji oraz dla szerokiego zakresu
aplikacji rozproszonych.
Korzystanie z serwisów wiąże się z opłatami, adekwatnymi do zużycia zasobów infrastruktury
informatycznej, na których jest wdrożona aplikacja.

Z analizy powyższych rozwiązań wynika, że na rynku istnieją wspierane rozwiązania PaaS,
jednak wszystkie rozwiązania wiążą się z opłatami za korzystanie z usługi.
Dodatkowo każde z rozwiązań skupia się przede wszystkim na usprawnianiu procesu wdrożenia
oprogramowania.
Z tych powodów potrzebna jest implementacja własnego systemu wspomagającego zarządzanie
wieloma kontenerami LXC, dostosowanego do potrzeb dydaktycznych przedmiotu Systemy Operacyjne.

\section{Założenia}
\label{sec:zalozenia}

Zaprojektowany system powinien spełniać następujące założenia:

\begin{itemize}
  \item System jest wielodostępny - może korzystać z niego wielu użytkowników jednocześnie.
  \item System zarządza kontenerami LXC wykonując polecenia powłoki na zdalnym serwerze, łącząc
  się do niego poprzez SSH\@.
  \item Każdy utworzony kontener powinien umożliwiać podłączenie się do niego za pomocą
  protokołu SSH\@.
  \item Każda operacja wykonywana przez aplikację na kontenerach LXC jest asynchroniczna.
  \item Informacje o zmianie statusu utworzonych w systemie asynchronicznych zadań są przesyłane
  do aplikacji klienckiej poprzez gniazda.
  \item Każdy użytkownik systemu ma przypisaną przynajmniej jedną rolę, z którymi wiążą się
  określone poziomy dostępu.
  Możliwe role to klient lub administrator.
  \item Każdy użytkownik w systemie może mieć przypisanych do swojego konta wiele kontenerów.
  Może też nie mieć przypisanego żadnego kontenera.
  \item Do uwierzytelnienia użytkownika w systemie należy podać poprawny login i hasło.
  Dane użytkownika potrzebne uwierzytelnienia są przechowywane w relacyjnej bazie danych.
  Hasło jest przechowywane w formie niejawnej.
  \item Bezstanowość - aplikacja serwerowa nie przechowuje stanu o sesji użytkowników.
  \item Autoryzacja użytkownika następuje prz każdym zapytaniu wysłanym do aplikacji serwerowej.
  \item Nieuwierzytelniony użytkownik ma dostęp jedynie do strony głównej na której
  może się uwierzytelnić.
  \item Rola administratora umożliwia zarządzanie kontenerami.
  Administrator może tworzyć i usuwać kontenery, zarządzać ich stanem oraz przypisywać je
  do użytkowników.
  Posiada także możliwość tworzenia nowych kont użytkowników, usuwania ich oraz zarządzania nimi,
  w tym ma możliwość zmiany hasła do kont użytkowników.
  Dodatkowo może przeglądać listę wszystkich zadań wykonanych na kontenerach przez wszystkich
  użytkowników systemu takich jak włączenie kontenera czy pobranie informacji o jego stanie.
  \item Rola klienta umożliwia zarządzanie stanem kontenerów przypisanych do uwierzytelnionego
  konta.
  Klient ma również możliwość zarządzania własnym kontem, w tym zmiany hasła.
  Dodatkowo może przeglądać listę zadań wykonanych na kontenerach przypisanych do jego konta.
  \item Pobrane dane o uruchomionym kontenerze powinny zawierać wszystkie dane potrzebne, aby
  podłączyć się do niego za pomocą protokołu SSH, z wyłączeniem danych wrażliwych tj.\ hasło.
  \item Aplikacja kliencka jest dostępna za pomocą przeglądarki internetowej i aktualizuje
  wyświetlane dane reaktywnie, bez potrzeby odświeżania lub ładowania nowej strony.
\end{itemize}

\section{Stos technologiczny}
\label{sec:stos technologiczny}

Do stworzenia ww.\ projektu wykorzystano dodatkowe narzędzia i technologie.
Ich licencje pozwalają na wykorzystanie ich w ramach pracy inżynierskiej.
Spis wszystkich użytych technologii wraz z wersjami i licencjami został przedstawiony
poniżej w tabeli~\ref{tab:technologie}.

\begin{table}[H]
  \centering
  \begin{tabular}{|c|c|c|}
    \hline
    Technologia & Wersja & Licencja \\
    \hline
    Apache Maven & 3.6.1 & Apache License 2.0~\cite{apachev2} \\
    \hline
    Docker & 19.03.1-ce & Apache License 2.0~\cite{apachev2} \\
    \hline
    Gson & 2.8.5 & Apache License 2.0~\cite{apachev2} \\
    \hline
    Guava & 27.1-jre & Apache License 2.0~\cite{apachev2} \\
    \hline
    Hibernate & 5.3.9.Final & LGPL 2.1~\cite{lgplv21} \\
    \hline
    JJWT & 0.9.1 & Apache License 2.0~\cite{apachev2} \\
    \hline
    LXD & 3.0.3 & Apache License 2.0~\cite{apachev2} \\
    \hline
    Material-UI & 3.5.1 & MIT~\cite{mit} \\
    \hline
    ModelMapper & 2.3.1 & Apache License 2.0~\cite{apachev2} \\
    \hline
    Node.js & 11.15.0 & MIT~\cite{mit} \\
    \hline
    Npm & 6.10.3 & Artistic License 2.0~\cite{artistic2} \\
    \hline
    OpenJDK & 11.0.1 & GPLv2~\cite{gplv2} \\
    \hline
    PostgreSQL & 12 Beta 3 & APostgreSQL License~\cite{postgreslicense} \\
    \hline
    Project Reactor & 3.2.10.RELEASE & Apache License 2.0~\cite{apachev2} \\
    \hline
    React & 16.7.0 & MIT~\cite{mit} \\
    \hline
    Spring Boot & 2.1.4.RELEASE & Apache License 2.0~\cite{apachev2} \\
    \hline
    Spring Data & 2.1.6.RELEASE & Apache License 2.0~\cite{apachev2} \\
    \hline
    Spring Security & 5.1.5.RELEASE & Apache License 2.0~\cite{apachev2} \\
    \hline
    Spring Web & 5.1.6.RELEASE & Apache License 2.0~\cite{apachev2} \\
    \hline
    SSHJ & 0.27.0 & Apache License 2.0~\cite{apachev2} \\
    \hline
  \end{tabular}
  \caption{Technologie użyte w ramach pracy inżynierskiej}
  \label{tab:technologie}
\end{table}

\clearpage