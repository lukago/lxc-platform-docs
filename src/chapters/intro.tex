\chapter{Wstęp}
\label{ch:wstęp}
\section{Cel}
\label{sec:cel}
Celem pracy jest zbadanie możliwości, jakie daje zestaw narzędzi Java EE
w zakresie tworzenia systemu internetowego~\cite{einstein},
na przykładzie internetowego systemu wspomagającego przeprowadzanie ankiet. \\
Wybór technologii został podyktowany szczegółową analizą popularnych języków programowania,
wnioski z której przedstawione zostały w tabeli~\ref{tab:porownanie_jezykow}.
Architektura tworzonego systemu przedstawiona jest na rysunku~\ref{fig:architektura_warstwowa}.

\begin{table}[H]
  \centering
  \begin{tabular}{|c|c|c|}
    \hline
    & Java & C++ \\
    \hline
    Opis & FAJNA& SŁABY \\
    \hline
  \end{tabular}
  \caption{Porównanie języków programowania.}
  \label{tab:porownanie_jezykow}
\end{table}

\begin{figure}[H]
  \centering \includegraphics[width=0.9\textwidth]{figures/layers-arch}
  \caption{Architektura warstwowa aplikacji.}
  \label{fig:architektura_warstwowa}
\end{figure}

\cleardoublepage

\subsection{Podsekcja}
\label{subsec:podsekcja}
Cel udało się osiągnąć przy użyciu programu, którego kod przedstawiony jest poniżej.

\begin{lstlisting}[caption={Wyświetlenie "SPOKO!"},captionpos=b]
  System.out.println("SPOKO!");
\end{lstlisting}